
%
% Stel je wilt het C++-programma iets.cc mooi printen,
% en wellicht er nog wat begeleidende tekst bij schrijven.
%

\documentclass{article}

\setlength{\textheight}{25.7cm}
\setlength{\textwidth}{16cm}
\setlength{\unitlength}{1mm}
\setlength{\topskip}{2.5truecm}
\topmargin 260mm \advance \topmargin -\textheight 
\divide \topmargin by 2 \advance \topmargin -1in 
\headheight 0pt \headsep 0pt \leftmargin 210mm \advance
\leftmargin -\textwidth 
\divide \leftmargin by 2 \advance \leftmargin -1in 
\oddsidemargin \leftmargin \evensidemargin \leftmargin
\parindent=12pt

\frenchspacing
\newcommand{\SA}{Simulated Annealing }

\usepackage[english]{babel}
\usepackage{amsmath}
\usepackage{float}
\usepackage{graphicx}
\restylefloat{table}

\usepackage{listings}
% Er zijn talloze parameters ...
\lstset{language=C++, showstringspaces=false, basicstyle=\small,
  numbers=left, numberstyle=\tiny, numberfirstline=false,
  stepnumber=1, tabsize=4, 
  commentstyle=\ttfamily, identifierstyle=\ttfamily,
  stringstyle=\itshape}

\title{Neural Networks: Assignment 1}
\author{Pepijn van Heiningen \& Michiel Vos}

\begin{document}

\maketitle

\section{Introduction}
For the first assignment of the Neural Networks course there were two tasks:
\begin{itemize}
\item Parameter Estimation
\item Classification of handwritten digits
\end{itemize}

\subsection{Task 1}
The first task consists of $3$ different parts. For the first part we were given four samples and four density functions. The first task was to find the most likely density function that was used to generate a sample. 
The second part asked to demonstrate that the area under the parabolic density function was always equal to $1$, indepentendly on the value of the input parameter \verb+s+.
Finally we were given a dataset which corresponded to a Gaussian mixture of two normal distributions. The assignment was to find the correct values for the parameters using three different methods.

\subsection{Task 2}

\section{Task 1: Parameter Estimation}
\subsection{Problem Description}
We were given 4 datasets (A, B, C and D) and 4 density functions (Normal, Parabolic, Triangular and Rectangular). The task was to find out what dataset was most likely generated by a particular density function. 

Subsequently we had to demonstrate that the area under the parabola was always $1$.

Finally we were given another dataset, a Gaussian mixture of two normal distributions. We were asked to estimate the six different parameters ($p_A, p_B, \mu_A, \mu_B, \sigma_A, \sigma_B$) using three different methods:

\begin{itemize}
\item By directly calulating the values of the parameters.
\item By trying to brute-force the correct values.
\item By using the Expectation Maximization algorithm.
\end{itemize} 
\newpage
\subsection{Problem Solution}
To solve the first part of task $1$, we wrote a function that estimated the log likelihood that a sample came from a given distribution. 
The log-likelihood can be calculated by:\\
\[\sum_{x \in X} \ln {p(x|\theta)}\]
We applied it to all four samples, ranging the parameter \verb+s+ between $0,5$ and $1,5$ with a stepsize of $0,001$, to find the most likely distribution type. In table \ref{table:data1} you can find the results. \\

\begin{table}[H]
	\begin{center}
		\begin{tabular}{l|l|l}
			Dataset & Density function & \verb+s+ \\
			\hline
			\verb+A+ & Parabolic   & 0,79  \\
			\verb+B+ & Normal  & 0,54 \\
			\verb+C+ & Rectangular   & 1,09   \\
			\verb+D+ & Triangular   & 0,99   \\
		\end{tabular}
		\caption{Datasets and most likely density functions}
		\label{table:data1}
	\end{center}
\end{table}

To demonstrate that the area under the parabolic density function was always $1$, we calculated the area under the curve for input values of \verb+s+ between $0,01$ and $100$ with a stepsize of $0,01$. By showing that the mean of these values is $1$ and the standard deviation $0$, we know that all output values were equal to $1$.\\

For the third part we calculated the six parameters directly by using the class variable. The results are in table \ref{table:data2}. \\\\

\begin{table}[H]
	\begin{center}
		\begin{tabular}{l|l|l|l|l|l}
			$p_A$ & $p_B$ & $\mu_A$ & $\mu_B$ & $\sigma_A$ & $\sigma_B$ \\
			\hline
			0,63 & 0,37 & 46,81 & 63,63 & 3,67 & 1,18 \\
		\end{tabular}
		\caption{Calculated parameters and their values}
		\label{table:data2}
	\end{center}
\end{table}

Subsequently we tried the brute-force technique to find the optimal values. The number of combinations was: $2,25 \cdot 10^{11}$. In total it would take $5,77 \cdot 10^6$ seconds or approximately $66,8$ days to brute-force on an Intel Core $i5$-$3470$. Of course we didn't run the entire algorithm, but only measured $10000$ iterations. Due to overhead the final runtime might be different, but we do not have the time to test it.\\\\

Finally we used the EM-algorithm to find the optimal values of the mixture parameters. (See table \ref{table:data3}) First we initialized the Gaussian mixture model by using the k-means algorithm for $10$ iterations. Then we trained it using the given dataset for another $10$ generations. Training the model for $10$ iterations took  $0.004762$ seconds, so this approach is approximately $4.7 \cdot 10^{13}$ times faster than the brute-force approach. If we compare the values found with the ``true'' model, we see that there are no differences to $2$ decimals behind the comma. 

\begin{table}[!h]
	\begin{center}
		\begin{tabular}{l|l|l|l|l|l}
			$p_A$ & $p_B$ & $\mu_A$ & $\mu_B$ & $\sigma_A$ & $\sigma_B$ \\
			\hline
			0,63 & 0,37 & 46,81 & 63,63 & 3,67 & 1,18 \\
		\end{tabular}
		\caption{Approximated parameters and their values}
		\label{table:data3}
	\end{center}
\end{table}


\section{Task 2: Classification of handwritten digits}
\subsection{Problem Description}
The goal of task $2$ is to recognize hand-written digits. There are two sets available with images of the hand-written digits, with known labels. So, this is a supervised learning problem. The first set is the training set. With this data we are going to make a model and then model can be checked with the testset. If the model performs well on the test set, it indicates the model isn't overfit or overtrained. There are multiple solutions to this problem.

\subsection{Problem Solutions}
\subsubsection{Distance-based classifier}
An image consists of an image of $16$x$16$ pixels. Those points or pixels have a value in greyscale. Based on these values we can say something about an image, a so called feature. First we load all the images of digit $0$ of the training set and extract the features. Then we can calculate the average values. We repeat this for every digit. For an image from the testset, where we don't know the label of, we extract also the features and compare them with the values of the averages we calculate in the step before. The one with the least distance will be probably the correct digit.

The features we use are the center, the diameter, the radius and the standard deviation of distances from the center, but more and better features can be implemented. 

When the center feature is used for training the model and the trainingset is checked, the accuracy is $86\%$. When the model is checked on the test set, the accuracy is lowered to $80\%$, but this is expected to be lower. See table \ref{tab:cm} for the confusion matrix. The numbers on the diagonal from top left to bottom right are the correctly classified digits. Most digits are on that line, but it's not perfect and there are errors. Digit $0$ and digit $6$ are often confused. 

\begin{table}[H]
	\begin{center}
		\begin{tabular}{llllllllll}
				178 & 0 & 2 & 3 & 1 & 3 & 7 & 0 & 3 & 0 \\
				0 & 120 & 0 & 0 & 3 & 0 & 0 & 2 & 2 & 5 \\
				3 & 0 & 69 & 3 & 3 & 0 & 2 & 1 & 0 & 0 \\
				2 & 0 & 6 & 61 & 0 & 6 & 0 & 0 & 6 & 0 \\
				4 & 0 & 8 & 1 & 69 & 3 & 2 & 5 & 3 & 8 \\
				2 & 0 & 1 & 8 & 0 & 38 & 1 & 0 & 3 0 \\
				23 & 1 & 0 & 0 & 1 & 1 & 78 & 0 & 0 & 0 \\
				1 & 0 & 2 & 0 & 1 & 0 & 0 & 50 & 0 & 5 \\
				10 & 0 & 13 & 1 & 0 & 0 & 0 & 0 & 73 & 2 \\
				1 & 0 & 0 & 2 & 8 & 4 & 0 & 6 & 2 & 68 \\
		\end{tabular}
	\end{center}
	\caption{Confusion Matrix}
	\label{tab:cm}
\end{table}

\subsubsection{Perceptron Algorithm}
A perceptron is a small neural network. It's based on inputs, weights and outputs. The inputs are the pixels of a image. With the correct weight it's possible to predict the right output. The weights gives importance to a specific pixel. In our case there is just one output node which can have two values. $0$ or $1$, or true and false. We built a perceptron with $10$ nodes. Each node $x$ answers the question ``Is this hand-written image digit $x$?'' So after checking an image with each of the $10$ nodes, we should be able to classify the input data.\\

On the trainingset the perceptron achieves a perfect accuracy, as proven by the Covers theorem. On the trainingset the trained perceptron reaches an accuracy of about $87$\%. If we look at the errors it makes, we see that digit $8$ is sometimes classified as a $3$. Also $9$ and $4$ are hard to separate.\\

Training the perceptron takes about $60$ seconds. The perceptron is trained with the trainingset, which is forwarded through the net for $100$ iterations. You can see the error decreasing in figure \ref{fig:perceptronerror}.\\

\begin{figure}
	\centering
		\includegraphics[width=\textwidth]{D:/Universiteit/NN/nn1.2/perceptronerror.png}
	\vspace{-0.2in}
	\caption{Error decrease of perceptron}
	\label{fig:perceptronerror}
\vspace{-1in}
\end{figure}

The models make errors on the testset, but it can be proven that a perceptron always converges on linearly separable data. With the perceptron other errors can be made as well. The Distance-based classifier can classify a digit wrong, but with the perceptron it can happen that the a digit could not be classified or a digit can be classified as multiple classes. This makes a comparion between the confusion matrices difficult. 

\subsubsection{Logistic regression classifier}
For the logistic regression classifier we used the built in function from Netlab. As described in the assignment we built a network with a similar topology as the perceptron. Again each node classifies only $1$ digit.\\
Compared to the perceptron the logistic regression classifier produced less accurate results, but it was a lot faster than the perceptron algorithm. On the trainingset an accuracy of $89$ percent was achieved, lowering to $75$\% on the testset. All digits that are classified are classified correctly, but not all digits are classified.

\subsubsection{Alternative representations}
In this part there is experimented with different inputs. From a image we extract $6$ more features:

\begin{itemize}
\item The amount of white pixels in a image. These values has the exact value of $1$. Some digits need more digital ink compared to others, the reason why it could be a good feature for separating digits.
\item We do the same for the black pixels.
\item And the gray pixels.
\item The height could also be a good feature to classify. The height is defined as the index of the row of the first white pixel minus the index of the row of the last white pixel plus $1$. *The value of a white pixel is exact $1$, not $0.99$. 
\item The width is calculated the nearly the same. This feature might even be better, because the averages are more spread.
\item The average color value of a image. 
\end{itemize}

See table \ref{tab:accuracy} for the statistics. Though the features do not produce the best results, the results can be used to compare the features.. The amount of black pixels predicted around $30$\% of the testset correctly. That is strange, because the black pixel feature could only predict $22$\%  correctly and we think there should be a correlation between black and white pixels. Width is better than height. That is explainable, because if we humans look at the $10$ digits, we see that the height are all around the same and the width is not. Combining several simple features would be a good idea to predict digits. 

\begin{table}[H]
	\begin{center}
	\begin{tabular}{l|l}
		Feature & Accuracy \\
		\hline
		Black & 30.1\% \\
		White & 22.4\% \\
		Gray & 22.9\% \\
		Height & 13.1\% \\
		Width & 34.4\% \\
		Color & 30.5\% \\
	\end{tabular}
	\end{center}	
	\caption{Accuracy of the features}
	\label{tab:accuracy}
\end{table}

If input these as extra features into the logistic regression classifier, we achieve better results than before. 89\% on the trainingset and 90\% on the testset.

\end{document}
